\chapter{Introduction}

\section{Background}

It is generally useful for software to be able to programmatically determine
the capabilities of the hardware it is running on. External debuggers need to
know the same information. RISC-V extensions mostly use CSRs for this
information, but that is not flexible enough for some kinds of data like
hardware breakpoint capabilities, and cache hierarchy.

This document specifies syntax, semantics, discovery method, and access method
for a static data structure that can accommodate implementation parameters of
RISC-V standards beyond what can be easily put into CSRs. The structure is
called a Configuration Structure (CS).

\section{Goals}

\begin{steps}{The task group will deliver:}
\item A specification for a machine-readable format for the configuration
    structure. It's intended to be easily accessible by M-mode software and
    debuggers.
\item A specification for a human-readable format for the configuration
    structure. It's intended to be used to talk about the system description in
    this document as well as other documentation, to help designers write a
    configuration structure, and to display the configuration structure to
    people.
\item A software tool that translates back and forth between the
    machine-readable and human-readable format.
\item A specification for a method to discover and access, from M-mode
    software, the machine-readable configuration structures.
\end{steps}

The configuration structure should be flexible enough that future task groups
won’t feel the need to create another structure used to describe implementation
parameters. Implementation parameters are details that a RISC-V specification
explicitly leaves up to an implementation. This includes hart-specific details
like the kinds of hardware triggers supported, as well as details that are
outside harts such as the supported abstract debug commands.

\section{Use Cases}

\subsection{External Debuggers}

When an external debugger connects to a system, it would like to discover as
much as possible about that system in as little time as possible. Some of this
is merely to show the user (e.g. a manufacturer name), while other features are
critical to the user (e.g. XLEN), and other features determine what kind of
operations the user can perform (e.g. supported hardware trigger types). Most
of these are already discoverable, although many require writing a value and
checking the result to see whether support exists.

Any structure that's accessible from M-mode software will already be
accessible by the debugger. There might be a structure embedded in the Debug
Module itself which is only accessible by the debugger.

The debug feature that is most complex to describe is hardware triggers. Each
hart may have billions of triggers (although 4 is more typical). Each of those
triggers can be one of 4 types, and each type has its own options. Options
are things like trigger on execute/load/store, in M/S/U mode, chain to other
trigger, exact/greater/less-than value match, etc. It's permissible for
features to be implemented, but not in all combinations. E.g. greater value
might work in combination with load/store, but not together with executed.
Each trigger is configured by writing an XLEN-bit register.

A simple way to represent this information, without worrying about every
option individually, is to think of each trigger as its numeric value. Many
values can be compactly described by a list of value/mask tuples, low/high
tuples, or low/high/mask triples. Most implementations can get away with just
one or two tuples/triples. The alternative is to have some language that
describes logical expressions which feels much more complex.

It will be common for some or all triggers to support the same set of
features, so it's important to be able to group them together. Furthermore,
it will be common for multiple harts to support the same set of triggers. We
should be able to combine the trigger information between them, assuming
there is one structure that describes multiple harts. Putting that all
together, here's an example of what we'd want to describe:

\begin{itemize}
    \item Hart 0 triggers:
          \begin{itemize}
              \item Triggers 0--3:
                    \begin{itemize}
                        \item triple of LOW, HIGH, MASK
                    \end{itemize}
              \item Trigger 4:
                    \begin{itemize}
                        \item tuple of VALUE0, MASK0
                        \item tuple of VALUE1, MASK1
                    \end{itemize}
          \end{itemize}
    \item Hart 1--4 triggers:
          \begin{itemize}
              \item Triggers 0--1:
                    \begin{itemize}
                        \item triple of VALUE0, MASK0
                        \item triple of VALUE1, MASK0
                    \end{itemize}
          \end{itemize}
\end{itemize}

In addition there are abstract commands, which have similar issues. There are
a few commands, with a number of options. Tuples/triples as described above
would work. An example description might be:

\begin{itemize}
    \item Debug Module Abstract Commands
          \begin{itemize}
              \item triple of LOW, HIGH, MASK
              \item tuple of VALUE0, MASK0
              \item tuple of VALUE1, MASK1
              \item \dots
          \end{itemize}
\end{itemize}

\subsection{System Firmware}

Typical system firmware is executed when the system is powered on. It
initializes hardware, and builds up firmware services or data structures for
booting up system to OS. Examples are uboot for embedded systems, and BIOS for
PCs.

Through a combination of checking CSRs and
accessing the system description (Section~\ref{sec:AccessMethod}), firmware
can programmatically determine the hardware capabilities and configure hardware
accordingly.  These hardware capabilities can include availability and
implemented features of Physical Memory Protection (PMP), Core Local Interrupt
Controller (CLIC), Core Local Interruptor (CLINT), memory map, Real Time Clock (RTC), reset mechanism, and any future optional core features.

The Configuration Structure is an efficient alternative to testing for specific
hardware features (including handling failures) or customizing system firmware
for the specific system it will run on.

Often system firmware will take the information it has learned from the system
description as well as through other methods, and encode it into a different
industry-standard data structure (like Devicetree).  This structure is then
passed to the subsequent boot process.

\begin{commentary}
The system firmware mentioned refers to startup software which is executed
when the system powers on. The system firmware initializes hardware configuration and
builds up firmware services or data structures for booting up system to OS. The typical
system firmware such as uboot for embedded systems, BIOS for PCs or other firmware
frameworks.
\end{commentary}

\subsubsection{RISC-V Hart Hardware Features Use Cases (Structure format is TBD)}
\begin{itemize}
    \item Privilege Mode Capability

    Configuration structure returns the bitmaps of privilege modes supported on this hart.
    \item Base Integer ISA Width

    Configuration structure returns the base integer ISA register width.
    \item Physical Memory Protection (PMP) Availability

    Configuration structure returns one bit indicates if this hart supports PMP or not. if PMP is supported, M-mode PMP
    CSRs are implemented on this hart.
    \item Supervisor Mode Address Translation Modes Capability
    
    Configuration structure returns bitmaps indicate the supervisor mode address translation modes supported on this hart.
    
    \item Enhanced Physical Memory Protection (ePMP) Availability
    
    Configuration structure returns one bit indicates if ePMP is supported on this hart. If ePMP is supported,
    Machine Security Configuration is valid and specified in mseccfg M-mode CSR.
    \item Supervisor mode Physical Memory Protection (sPMP) Availability
    
    Configuration structure returns one bit indicates if sPMP is supported on this hart. If sPMP is supported,
    S-mode Physical Memory Protection CSRs are implemented on this hart.
\end{itemize}

\subsubsection{RISC-V Core Hardware Features Use Cases (Structure format is TBD)}
\begin{itemize}
    \item Core Local Interrupter Availability
    \begin{itemize}    
        \item Machine Mode Time Register Address of this hart
        \item Machine Mode Time Compare Register Address of this hart        
    \end{itemize}
    \item Core Local Interrupt Controller (CLIC) Availability
    
    Configuration structure returns one bit indicates if this hart supports CLIC or not. If CLIC is supported,
    the base address of CLIC memory-mapped registers is specified in M-mode mclicbase M-mode CSR.
    
\end{itemize}
\subsubsection{RISC-V Multicore Hardware Features Use Cases}

TBD

\chapter{Machine-Readable Format}

\section{Considerations}

Software in early bootup stages might want to parse some of the format. This
limits the complexity of the parsing process. For reference,
Table~\ref{tab:earlyresources} lists the resources available to software when
the first updateable RISC-V instruction executes in some real-world platforms.

\begin{table}[H]
    \centering
    \caption{Resources available to software when the first updateable RISC-V
    instruction executes in some real-world platforms.}
    \label{tab:earlyresources}
    \begin{tabular}{|l|l|l|l}
        \hline
        Platform & RAM & Flash/ROM & Clock Speed \\
        \hline
        HiFive1 & 16 kB & 8 kB & 14.4 MHz \\
        \hline
        HiFive1 Rev B & 16 kB & 4 MB & 14.4 MHz \\
        \hline
        HiFive Unleashed & 2 MB L2-LIM & 32 MB & 33.3 MHz \\
        \hline
        Intel Whitley & stackless code? & 32MB & probably 3.7GHz \\
        \hline
        ET-SvcProc & 1 MB & 128 KB & 10 MHz \\
        \hline
        ET-AppProcs & 256 KB + cache & 0 KB & 100 MHz+ \\
        \hline
        OpenTitan (small option) & 64 KB & 512 KB & \\
        \hline
        GigaDevice & 32 KB & up to 128 KB & 108 MHz \\
        \hline
    \end{tabular}
\end{table}

\section{Misc}

\begin{steps}{Ideas:}
\item Fully custom, along the lines of
    \url{https://github.com/riscv/riscv-debug-spec/pull/450/files#diff-9cee87c1f0feb10d9e3e9b2bfad92985R380}
\item Semi-custom, using CBOR
\item ...
\end{steps}

\chapter{Human-Readable Format}

\begin{steps}{Ideas:}
\item Take the binary format, and expand constants to strings, add whitespace,
keeping the mapping as close to 1:1 as possible.
\item XML
\item JSON
\item ...
\end{steps}

\chapter{Binary Format}

\begin{steps}{We're investigating 2 formats:}
    \item Protobuf, which is well documented at \url{https://developers.google.com/protocol-buffers}
    \item A custom format.
\end{steps}

\section{Custom Format}

\subsection{Bit Stream}

The custom format is a stream of bits, turned into a stream by filling in
bytes from the top. The first bit in the stream is found in the most
significant bit of the first byte. The second bit in the stream is found in
the next most significant bit of the first byte, and so on. When 8 bits have
been put in a byte, work on the next byte starts. Table \ref{tab:bitorder}
shows exactly where the first 32 bits of a stream end up.

\begin{table}[H]
    \centering
    \caption{How 32 bits are stuffed into bytes.}
    \label{tab:bitorder}
    \begin{tabular}{|r|r|r|r|r|r|r|r|r|}
        \hline
        & 7 (MSB) & 6 & 5 & 4 & 3 & 2 & 1 & 0 (LSB) \\
        \hline
        byte 0 & 0 & 1 & 2 & 3 & 4 & 5 & 6 & 7 \\
        byte 1 & 8 & 9 & 10 & 11 & 12 & 13 & 14 & 15 \\
        byte 2 & 16 & 17 & 18 & 19 & 20 & 21 & 22 & 23 \\
        byte 3 & 24 & 25 & 26 & 27 & 28 & 29 & 30 & 31 \\
        \hline
    \end{tabular}
\end{table}

\subsection{Variable Length Numbers (VLN)}

Variable Length Numbers (VLN) are very common in the binary format. They're
encoded based on a sequence of lengths defined in Table \ref{tab:vln}. Each
number is encoded as a fixed number of bits (the amount of bits is taken from
the table), follow by a single bit which indicates if there are more bits for
the number.

\begin{table}[H]
    \centering
    \caption{VLN runs}
    \label{tab:vln}
    \begin{tabular}{|r|r|r|r|r|r|r|r|r|r|}
        \hline
        3 & 4 & 5 & 7 & 10 & 14 & 19 & 25 & 32 & 40 \\
        \hline
    \end{tabular}
\end{table}

\begin{steps}{To decode a VLN:}
    \item Get $N$ from the next value in Table \ref{tab:vln}.
    \item Read $N$ bits from the bit stream into $V$, where the first bit in
        the stream is the MSB of $V$.
    \item Extend the bits in the value read so far with $V$.
    \item Read another bit from the bit stream. If it is 0, then terminate.
    \item Repeat.
\end{steps}

\subsection{Types}

\subsubsection{Number}

A number that is encoded as a single VLN.

\subsubsection{Boolean}

A value encoded as a single bit. 1 means true, and 0 means false.

\subsubsection{Custom}

\begin{steps}{Custom types are are described in JSON/YAML format, and consist
of 1 or more entries. Each entry consists of:}
\item A human-readable name.
\item A numeric code, which is used to identify this entry in the binary format.
\item The type of the entry.
\item A required flag, defaulting to false. If it's true, then the entry must
be present in the binary. This can provide large savings in the binary size
when the value is small such as a boolean.
\item A repeatable flag, defaulting to false. If it's true, then there may 
be more than one value associated with this entry.
\end{steps}

When a custom type is encoded, first the length of the encoded type
(excluding this length itself) is added to the bitstream, unless each entry
is required.

Next come all the required entries, in the order of their code (lowest
first).

Then come the optional entries. Each optional entry starts with its code
number, always followed by the size of the encoded entry, and then the entry
value itself.

\subsubsection{Repeatable}

If a type is repeatable, then the first value in the stream is the total
length of all the repeated values combined, followed by each individual value
encoded as described above.

\chapter{Access Method}
\label{sec:AccessMethod}

The binary configuration structure is memory-mapped in system memory. There
is a CSR which contains the physical address where the structure starts.

\begin{steps}{Ideas:}
\item A CSR contains the base address of where the structure can be accessed on
the system bus
\item A CSR address/data pair can be used to access the structure. (Write index
to the address CSR, then read from the data CSR.)
\item ...
\end{steps}